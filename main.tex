%
%%%%%%%%%%%%%%%%%%%%%%%%%%%%%%%%%%%%%%%%%%%%%%%%%%%%%%
%                                                    %
%     Modelo para Trabalho de Conclusao de Curso     %
%                                                    %
%                   Template LaTeX                   %
%                                                    %
% Elaboracao  : Grupo PET-Tele                       %
%                                                    %
% Responsaveis:                                      %
%               Marcio Camoleze de Andrade (2008)    %
%               Thiago Muniz de Souza (2008)         %
%                                                    %
% Orientacao  : Prof. Alexandre Santos de la Vega    %
%                                                    %
% Versoes:                                           %
%   - dez/2021 (atualizacao em revisao)              %
%   - set/2017 (primeira revisao estavel)            %
%   - abr/2008 (primeira versao  estavel)            %
%                                                    %
%%%%%%%%%%%%%%%%%%%%%%%%%%%%%%%%%%%%%%%%%%%%%%%%%%%%%%
%

%
\documentclass[12pt,a4paper,oneside]{book}
%
%
%%%%%%%%%%%%%%%%%%%%%%%%%%%%%%%%%%%%%%%%%%%%%%%%%%%%%%
%                Inclusao de pacotes                 %
%%%%%%%%%%%%%%%%%%%%%%%%%%%%%%%%%%%%%%%%%%%%%%%%%%%%%%
%
%
%
%%%%%%%%%%%%%%%%%%%%
%   Pacotes gerais %
%%%%%%%%%%%%%%%%%%%%
%
%
% Padrao brasileiro:
%   a primeira frase eh indentada
%   para todos os paragrafos.
%
\usepackage{indentfirst}
%
%
% Pacotes para lingua portuguesa
%
%%%\usepackage{babel}
%%%\usepackage[brazil]{babel}
\usepackage[brazilian]{babel}
%%%\usepackage[portuguese]{babel}
%%%\babelprovide[import=pt-BR,main]{portuguese}
%
%
%%%\usepackage{ae}
\usepackage{lmodern}
\usepackage[T1]{fontenc}
%
%
% Padrao de codificacao 
% dos caracteres no arquivo '.tex'
%
%%%\usepackage{inputenc}
%\usepackage[ansinew]{inputenc}
\usepackage[utf8]{inputenc}
%
%
% Pacotes matematicos
%
% Package dependency hierarchy 
% in terms of the AMS-LaTeX bundle:
%
% - amsmath (miscellaneous enhancements)
%   - amstext (text embedded in mathematics)
%     - amsgen (*)
%
% - amsbsy (bold symbols)
%   - amsgen (*)
%
% - amsopn (operator name commands)
%   - amsgen (*)
%
% - amssymb (extended symbol collection)
%   - amsfonts (*)
%
% - amsthm (theorem-like environments)
%
% (*) This package has no dependencies.
%
\usepackage{amsmath}
\usepackage{amsfonts}
\usepackage{amssymb}

%
%
% Pacotes graficos
%
%\usepackage{graphics}
\usepackage{graphicx}
\usepackage{subfig}
%\usepackage{subfigure}
\usepackage{epsfig}
%\usepackage{rotating}
%
%
% Pacotes sobre tabelas
%
\usepackage{multirow}
%
%
% Pacote para fazer 'Indice Remissivo'
%
%%%\usepackage{makeidx}
\usepackage{imakeidx}
\makeindex[intoc]
%
%
% Pacotes uteis para revisao do texto
%
% Comentarios de varias linhas
%
\usepackage{comment}
%
%
%%%%%%%%%%%%%%%%%%%%%
% Incluir Códigos   %
%%%%%%%%%%%%%%%%%%%%%
\usepackage{listings}
%
%% Configuracoes do pacote listings
\lstdefinestyle{mystyle}{
language=Python,
basicstyle=\ttm,
% Add keywords here
morekeywords={self, if, True, False, for, except, raise, return, try},
keywordstyle=\ttb\color{deepblue},
% Custom highlighting
emph={set_mode, set_servo_pulsewidth, sleep, OUTPUT, stop, exit},          
% Custom highlighting style
emphstyle=\ttb\color{deepred},    
stringstyle=\color{deepgreen},
% Any extra options here
frame=single,                       
showstringspaces=false,
commentstyle=\color{green}
}
% 
\lstset{
style=mystyle,
breaklines=true
}
%
%
% definicao de comando para Lista de Códigos
%
% Listing -> Código
\renewcommand{\lstlistingname}{Código}
% List of Listings -> Lista de Códigos
\renewcommand{\lstlistlistingname}{Lista de \lstlistingname s}
%
%% PACOTES QUE SERÃO REMOVIDOS APÓS REVISÃO
% Realce de texto
\usepackage{soulutf8}
\usepackage{soul}
% Extensão de letras gregas (Vou tentar Excluir isso)
\usepackage{upgreek}

%%%%%%%%%%%%%%%%% FIM DESSES PACOTES %%%%%%%%%%%%%%

% Marcacao de texto com cor
%
% Explicar...
%
%%%\usepackage{color}
%\usepackage{xcolor}
\usepackage[table,xcdraw]{xcolor}
%
\definecolor{red}  {rgb}{1,0,0}
\definecolor{green}{rgb}{0,1,0}
\definecolor{blue} {rgb}{0,0,1}
\definecolor{deepblue}{rgb}{0,0,0.5}
\definecolor{deepred}{rgb}{0.6,0,0}
\definecolor{deepgreen}{rgb}{0,0.5,0}
\definecolor{green}{rgb}{0,0.8,0}
%
\DeclareFixedFont{\ttb}{T1}{txtt}{bx}{n}{12} % for bold
\DeclareFixedFont{\ttm}{T1}{txtt}{m}{n}{12}  % for normal
%
%
% O pacote 'xurl' 
% faz quebra automatica de linha 
% para um URL muito grande...
%
%%%\usepackage{url}
\usepackage{xurl}
%
\usepackage{hyperref}
%
\usepackage{float}
%
\usepackage{enumitem}
%
\usepackage{cleveref}
% Explicar e posicionar...
%
%%%\usepackage{memhfixc}
%

%%%%%%%%%%%%%%%%%%%%%%%%%%%%%%%%%%%%%%%%%%%%%%%%%%%%%%
%
%
%%%%%%%%%%%%%%%%%%%%%%%%%%%%%%%%%%%%%%%%%%%%%%%%%%%%%%
%              Formatacao da Pagina                  %
%%%%%%%%%%%%%%%%%%%%%%%%%%%%%%%%%%%%%%%%%%%%%%%%%%%%%%
%
%%%%%%%%%%%%%%%%%%%%%%%%%%%%%%%%%%%%%%%%%%%%%%%%%%%%%%
%              Formatacao da Pagina                  %
%%%%%%%%%%%%%%%%%%%%%%%%%%%%%%%%%%%%%%%%%%%%%%%%%%%%%%

%
% Paper size A4: width=210mm ; height=297mm
%

% horizontal
\setlength{\hoffset}{-1in}

\setlength{\oddsidemargin}{3.0cm} 

\setlength{\textwidth}{160mm}  % (210mm - 30mm - 20mm)

\setlength{\parindent}{1.25cm} % indentacao de cada paragrafo

% vertical
\setlength{\voffset}{-1in}
\addtolength{\voffset}{2.0cm}

\setlength{\topmargin}{0.0cm}

\setlength{\headheight}{5mm}
\setlength{\headsep}{5mm}

\setlength{\textheight}{247mm} % (297mm - 30mm - 20mm)

%%%\setlength{\footskip}{0mm}

%%%%%%%%%%%%%%%%%%%%%%%%%%%%%%%%%%%%%%%%%%%%%%%%%%%%%%
%
%
%%%%%%%%%%%%%%%%%%%%%%%%%%%%%%%%%%%%%%%%%%%%%%%%%%%%%%
%                     Definicoes                     %
%%%%%%%%%%%%%%%%%%%%%%%%%%%%%%%%%%%%%%%%%%%%%%%%%%%%%%
%
% espacamento entre linhas: 
%   \linespread{factor}
%   factor=1.0 (espaço simples)
%   factor=1.3 (espaço 1 1/2)
%   factor=1.6 (espaço duplo)
\linespread{1.3} 
%
\pagestyle{myheadings}
%
\makeindex

%%%%%%%%%%%%%%%%%%%%%%%%%%%%%%%%%%%%%%%%%%%%%%%%%%%%%%


%%%%%%%%%%%%%%%%%%%%%%%%%%%%%%%%%%%%%%%%%%%%%%%%%%%%%%
%             Regras de Divisao Silabica             %
%%%%%%%%%%%%%%%%%%%%%%%%%%%%%%%%%%%%%%%%%%%%%%%%%%%%%%

\hyphenation{tra-ba-lho cur-so En-ge-nhei-ro}

%%%%%%%%%%%%%%%%%%%%%%%%%%%%%%%%%%%%%%%%%%%%%%%%%%%%%%

%
\begin{document}
%

%%%%%%%%%%%%%%%%%%%%%%%%%%%%%%%%%%%%%%%%%%%%%%%%%%%%%%
%                  Capa da Monografia                %
%%%%%%%%%%%%%%%%%%%%%%%%%%%%%%%%%%%%%%%%%%%%%%%%%%%%%%

\begin{titlepage}
  \begin{center}
    \Large{\textsc{Universidade Federal Fluminense} \\
           \textsc{Escola de Engenharia} \\
           \textsc{Curso de Graduação em Engenharia de Telecomunicações} 
          }
    \par\vfill
    \LARGE{Lúcio Folly S. Zebendo\\e\\João Luiz de Amorim Pereira Neto}
    \par\vfill
    \LARGE{Aplica\c{c}\~{o}es de \textit{Drones} em Redes de Computadores: Utilização da plataforma \textit{Raspberry Pi} como computador de bordo.}
    \par\vfill
    \Large{Niterói -- RJ\\
    2022}
  \end{center}
\end{titlepage}

%%%%%%%%%%%%%%%%%%%%%%%%%%%%%%%%%%%%%%%%%%%%%%%%%%%%%%


%%%%%%%%%%%%%%%%%%%%%%%%%%%%%%%%%%%%%%%%%%%%%%%%%%%%%%
%                   Folha de Rosto                   %
%%%%%%%%%%%%%%%%%%%%%%%%%%%%%%%%%%%%%%%%%%%%%%%%%%%%%%

\begin{center}

Lúcio Folly S. Zebendo\\e\\João Luiz de Amorim P. Neto

\vfill

Aplica\c{c}\~{o}es de \textit{Drones} em Redes de Computadores: Utilização da plataforma \textit{Raspberry Pi} como computador de bordo.

\vspace{3.0cm}

\begin{flushright}
\begin{minipage}{0.55\textwidth}
%
Trabalho de Conclusão de Curso 
apresentado ao Curso de Graduação em Engenharia de Telecomunicações 
da Universidade Federal Fluminense, 
como requisito parcial para obtenção 
do Grau de Engenheiro de Telecomunicações. 
%
\end{minipage}
\end{flushright}

\vspace{3.0cm}

Orientador: Prof. Dr. Alexandre Santos de la Vega

Coorientador - Prof. Dr. Lauro Eduardo Kozovits
\vfill

Niterói -- RJ

2022

\end{center}

\pagebreak

%%%%%%%%%%%%%%%%%%%%%%%%%%%%%%%%%%%%%%%%%%%%%%%%%%%%%%


%%%%%%%%%%%%%%%%%%%%%%%%%%%%%%%%%%%%%%%%%%%%%%%%%%%%%%
%                Numeracao em romano                 %
%%%%%%%%%%%%%%%%%%%%%%%%%%%%%%%%%%%%%%%%%%%%%%%%%%%%%%

\pagenumbering{roman}
\setcounter{page}{2}

%%%%%%%%%%%%%%%%%%%%%%%%%%%%%%%%%%%%%%%%%%%%%%%%%%%%%%


%%%%%%%%%%%%%%%%%%%%%%%%%%%%%%%%%%%%%%%%%%%%%%%%%%%%%%%%
%	            Ficha Catalografica                    %
%%%%%%%%%%%%%%%%%%%%%%%%%%%%%%%%%%%%%%%%%%%%%%%%%%%%%%%%

% 
% Info from LaTeX:
%
% \vspace{\fill} in a paragraph will 
%   add the filling vertical space 
%   below the line in which it eventually appears.
%
%   \vspace*{\fill}: the space is never removed.
%
%
% \vfill ends the paragraph at the spot and 
%   adds the filling vertical space.
%

\vspace*{\fill}

  \begin{figure}[!ht]
    \centering
%
% comando para inserir arquivo de imagem
%
%    \includegraphics[width=0.7\textwidth]{FichaCatalografica.jpg}

%
% box inserida apenas para ilustrar posicao da Ficha Catalografica
%
    \framebox{
      \begin{minipage}{0.7\textwidth}
        \centering
        A figura referente ao arquivo 

        \vspace{1cm}
        
        \textit{FichaCatalografica.jpg} 

        \vspace{1cm}
        
        fornecido pela Biblioteca 

        \vspace{1cm}
        
        deverá aparecer aqui. 

        \vspace{1cm}
        
        \textbf{ATENÇÃO: Na versão impressa, 
                essa página deverá ficar 
                no verso da página anterior.}
      \end{minipage}
    } 
  \end{figure}
%
% box inserida apenas para ilustrar posicao da Ficha Catalografica
%

\vspace*{\fill}

\clearpage

%%%%%%%%%%%%%%%%%%%%%%%%%%%%%%%%%%%%%%%%%%%%%%%%%%%%%%


%%%%%%%%%%%%%%%%%%%%%%%%%%%%%%%%%%%%%%%%%%%%%%%%%%%%%%
%                 Folha de Aprovacao                 %
%%%%%%%%%%%%%%%%%%%%%%%%%%%%%%%%%%%%%%%%%%%%%%%%%%%%%%

\begin{center}

Lúcio Folly S. Zebendo\\e\\João Luiz de Amorim P. Neto

\vspace{1.0cm}

Aplica\c{c}\~{o}es de \textit{Drones} em Redes de Computadores: Utilização da plataforma \textit{Raspberry Pi} como computador de bordo.

\vspace{1.0cm}

\begin{flushright}
\begin{minipage}{0.55\textwidth}
%
Trabalho de Conclusão de Curso 
apresentado ao Curso de Graduação em Engenharia de Telecomunicações 
da Universidade Federal Fluminense, 
como requisito parcial para obtenção 
do Grau de Engenheiro de Telecomunicações. 
%
\end{minipage}
\end{flushright}

\vfill

\begin{flushleft}

Aprovada em DIA de MÊS de ANO.

\end{flushleft}

\vfill

BANCA EXAMINADORA

\vfill

\hrulefill \\
Prof. Dr. Alexandre S. de la Vega - Orientador\\
Universidade Federal Fluminense - UFF

\vfill

\hrulefill \\
Prof. Dr. Lauro Eduardo Kozovits - Co-Orientador\\
Universidade Federal Fluminense - UFF

\vfill

\hrulefill \\
Prof. Alexandre S. de la Vega\\
INSTITUIÇÃO

\vfill

\hrulefill \\
Prof. Lauro Eduardo Kozovits\\
INSTITUIÇÃO

\vfill

Niterói -- RJ

2022

\end{center}

\pagebreak

%%%%%%%%%%%%%%%%%%%%%%%%%%%%%%%%%%%%%%%%%%%%%%%%%%%%%%


%%%%%%%%%%%%%%%%%%%%%%%%%%%%%%%%%%%%%%%%%%%%%%%%%%%%
%            Resumo na lingua vernacula            %
%%%%%%%%%%%%%%%%%%%%%%%%%%%%%%%%%%%%%%%%%%%%%%%%%%%%

\chapter*{Resumo}
%
\addcontentsline{toc}{chapter}{Resumo}
%
\thispagestyle{myheadings}
%
Este trabalho é o fruto de um esforço conjunto entre vários discentes e docentes da UFF e iniciativa privada, todos com mútuo 
interesse por VANTs e suas aplicações. O projeto inicial era ideia da Equipe UFFO~\cite{url:equipeuffo} - construir um \textit{drone} 
para vigilância dos campi da UFF. O hardware a ser utilizado era um quadricóptero equipado com um sistema FPV de transmissão analógica. 
Posteriormente, houve-se o interesse de equipar um computador de bordo no \textit{drone} em questão, para fazer aplicações no campo 
da visão computacional com processamento em tempo de voo. Para isso foi utilizado como referência o material da comunidade~\cite{url:ardupilotdoc} 
sobre computadores de bordo~\cite{url:ardupilot-companioncomputers}.\\
%
Para a escolha do computador de bordo foram levados mais em conta fatores de custo e popularidade, com isso a plataforma 
\textit{Raspberry Pi}~\cite{url:raspberrypi} foi a selecionada para esse fim.\\
%
A partir da arquitetura proposta, várias aplicações além da visão computacional são possíveis. Como o computador de bordo em questão 
possui um sistema operacional com propósito geral baseado em \textit{debian}~\cite{url:debian}, além de outras capacidades de Hardware como portas 
\textit{GPIO} para acoplamento de sensores e atuadores e outros barramentos para conexão de periféricos, as possibilidades para esse hardware são muitas.\\
%
Com isso, a aplicação proposta nesse trabalho será a conexão desse computador de bordo à uma rede IP. Várias capacidades da plataforma 
\textit{Raspberry Pi} serão exploradas nos experimentos que sucederão, além disso, serão desenvolvidas aplicações para testar o conceito de 
\textit{Drones} em Redes de computadores.
%

\bigskip

\hl{ Palavras-chave: Raspberry Pi. Drones. VANT. ROS. Equipe UFFO. Robótica. Ardpilot. Pixhawk}

%%%%%%%%%%%%%%%%%%%%%%%%%%%%%%%%%%%%%%%%%%%%%%%%%%%%%%


%%%%%%%%%%%%%%%%%%%%%%%%%%%%%%%%%%%%%%%%%%%%%%%%%%%%%%
%              Resumo na lingua inglesa              %
%                      Abstract                      %
%%%%%%%%%%%%%%%%%%%%%%%%%%%%%%%%%%%%%%%%%%%%%%%%%%%%%%

\chapter*{Abstract}
%
\addcontentsline{toc}{chapter}{Abstract}
%
\thispagestyle{myheadings}
%
This part is destinated to the abstract of your monograph. 
%
It must be written in the vernacular language and 
in an idiom of great popularization 
(English, French, Spanish, for example). 
%
This part should be done at last, because just after finishing the work 
it will be possible an overall understanding of it. 
%
The abstract should not bring any further information, 
it is just the summary of the relevants aspects of the monograph, 
such as work gender, finality, methodology, results and conclusions. 
%
It must be written impersonally, 
to possess an extension from 150 to 500 words 
typed in simple space and in only one paragraph. 
%
It must be followed by the keywords of your monograph. 

\bigskip

Keywords: Monograph. LaTeX. Hints. 

\newpage

%%%%%%%%%%%%%%%%%%%%%%%%%%%%%%%%%%%%%%%%%%%%%%%%%%%%%%


%%%%%%%%%%%%%%%%%%%%%%%%%%%%%%%%%%%%%%%%%%%%%%%%%%%%%%%%
%	                 Dedicatoria                       %
%%%%%%%%%%%%%%%%%%%%%%%%%%%%%%%%%%%%%%%%%%%%%%%%%%%%%%%%

\begin{flushright}
 \begin{minipage}{0.5\textwidth}
  %
  % espaço do topo até o início da dedicatória 
  \vspace{17.0cm} 
  %
  Espaço reservado para a dedicatória.
  %
 \end{minipage}
\end{flushright}



%%%%%%%%%%%%%%%%%%%%%%%%%%%%%%%%%%%%%%%%%%%%%%%%%%%%%%


%%%%%%%%%%%%%%%%%%%%%%%%%%%%%%%%%%%%%%%%%%%%%%%%%%%%%%%%
%	                 Agradecimentos                    %
%%%%%%%%%%%%%%%%%%%%%%%%%%%%%%%%%%%%%%%%%%%%%%%%%%%%%%%%

\chapter*{Agradecimentos}
\addcontentsline{toc}{chapter}{Agradecimentos}

\thispagestyle{myheadings}
%
Espaço reservado para os agradecimentos.
%
Agradecemos ao Prof. Alexandre pela orientação durante a nossa estadia no grupo do PET-Tele. Através, 
da nossa participação no grupo foi possível termos um primeiro contato com as plataformas de desenvolvimento 
Arduino e \textit{Raspberry Pi}~\cite{url:raspberrypi}, além de outras atividades que contribuíram para nosso
desenvolvimento profissional.
%
Agradecemos ao Prof. Raul pelo acolhimento na \textit{Equipe UFFO} e pelo incentivo ao estudo e desenvolvimento 
de VANTs. 
%
Agradecemos ao Prof. Lauro Eduardo pelo direcionamento durante a elaboração desse projeto de TCC. 
%
Agradecemos à Equipe UFFO e especialmente ao Raphael Miranda, pelo apoio no desenvolvimento do drone quadcóptero 
utilizado nos experimentos desse trabalho. 
%
Agradecemos à Empresa 6DDrones e especialmente ao Fabio, pelo apoio com infraestrutura e equipamentos para o desenvolvimento 
desse projeto.
%
Agradecemos à comunidade Arducopter pelo grande esforço colaborativo em desenvolver varias das ferramentas utilizadas nesse projeto.
%
Agradeço à Larissa, minha namorada, por todo apoio, carinho e incentivo durante a minha graduação.
%
Agradecemos aos nossos familiares pela cobrança e incentivo nos estudos.
%
%%%%%%%%%%%%%%%%%%%%%%%%%%%%%%%%%%%%%%%%%%%%%%%%%%%%%%


%%%%%%%%%%%%%%%%%%%%%%%%%%%%%%%%%%%%%%%%%%%%%%%%%%%%%%%%
%                   Lista de Figuras                   %
%%%%%%%%%%%%%%%%%%%%%%%%%%%%%%%%%%%%%%%%%%%%%%%%%%%%%%%%
%
\listoffigures
%
\addcontentsline{toc}{chapter}{Lista de Figuras}
%
\thispagestyle{myheadings}
%
%%%%%%%%%%%%%%%%%%%%%%%%%%%%%%%%%%%%%%%%%%%%%%%%%%%%%%


%%%%%%%%%%%%%%%%%%%%%%%%%%%%%%%%%%%%%%%%%%%%%%%%%%%%%%%%
%                   Lista de Tabelas                   %
%%%%%%%%%%%%%%%%%%%%%%%%%%%%%%%%%%%%%%%%%%%%%%%%%%%%%%%%
%
\listoftables
%
\addcontentsline{toc}{chapter}{Lista de Tabelas}
%
\thispagestyle{myheadings}
%
%%%%%%%%%%%%%%%%%%%%%%%%%%%%%%%%%%%%%%%%%%%%%%%%%%%%%%


%%%%%%%%%%%%%%%%%%%%%%%%%%%%%%%%%%%%%%%%%%%%%%%%%%%%%%%%
%                       Sumario                        %
%%%%%%%%%%%%%%%%%%%%%%%%%%%%%%%%%%%%%%%%%%%%%%%%%%%%%%%%
%
\tableofcontents
%
\thispagestyle{myheadings}

\clearpage

%%%%%%%%%%%%%%%%%%%%%%%%%%%%%%%%%%%%%%%%%%%%%%%%%%%%%%


%%%%%%%%%%%%%%%%%%%%%%%%%%%%%%%%%%%%%%%%%%%%%%%%%%%%%%
%                Numeracao em arabico                %
%%%%%%%%%%%%%%%%%%%%%%%%%%%%%%%%%%%%%%%%%%%%%%%%%%%%%%
%
\pagenumbering{arabic}

%%%%%%%%%%%%%%%%%%%%%%%%%%%%%%%%%%%%%%%%%%%%%%%%%%%%%%


%%%%%%%%%%%%%%%%%%%%%%%%%%%%%%%%%%%%%%%%%%%%%%%%%%%%%%%%
%                        Texto                         %
%%%%%%%%%%%%%%%%%%%%%%%%%%%%%%%%%%%%%%%%%%%%%%%%%%%%%%%%

\chapter{Introdução}
%
% retira numeracao da pagina, conforme as normas de apresentacao.
\thispagestyle{empty} 
%
%
Atrelado à popularização da tecnologia, o barateamento dos \textit{drones} tornaram-nos capazes de serem usados para além do âmbito militar, 
ganhando espaço nas industrias cinematográficas, esportivas e do entretenimento, fora para o uso pessoal.\\
Somado à isso, com o barateamento e miniaturização dos componentes eletrônicos nas últimas décadas, surgiram diversas plataformas de 
difusão de estudos sobre possíveis aplicações é, portanto, uma consequência inevitável. Dessa forma, abre-se um leque de possibilidades 
para explorar as atividades que um VANT (Veículo Aéreo Não Tripulado) pode executar.\\



\section{Motivações}

O sentimento de interesse e empolgação pela tecnologia dos drones dos autores desse texto foram as forças primordiais para a concepção 
do trabalho que se segue. Tal interesse nasceu dentro da Universidade Federal Fluminense através da equipe UFFO~\cite{url:equipeuffo}, 
formada por discentes e docentes da universidade.\\
Com a experiência adquirida na confecção de um drone quadcóptero simples, descobriu-se a possibilidade de acoplar a este um 
Raspberry Pi~\cite{url:raspberrypi} como computador de bordo da aeronave, expandindo suas capacidades.\\
A partir desta ideia, foram realizados estudos para buscar os métodos e tecnologias mais recentes utilizados nessa configuração de drone. 
Felizmente, muitas dessas tecnologias utilizadas já estavam disponíveis na documentação Open Source do Ardupilot~\cite{url:ardupilotdoc} 
para o equipamento utilizado.

\section{Objetivo}

O primeiro objetivo deste trabalho é projetar, construir e testar uma conexão de  um drone a uma rede \textit{IP}, visando facilitar a coleta de dados e o controle do drone em si. Para isso, no ambiente simulado, utilizaremos uma rede \textit{wi-fi} para realizar o envio de dados do drone a um servidor remoto, cujo intuito é armazenar, processar e exibir informações coletadas pelo drone. O segundo objetivo, agora fora do ambiente de simulação, gira em torno do estudo da viabilidade e benefícios 
de uma conexão utilizando a tecnologia LTE. ampliando o ambiente de funcionamento para muito além do alcance do \textit{wi-fi}.
\textit{IP}.  

\section{Resultados esperados}

\begin{itemize}
\item Criação de um ambiente de simulação para a validação dos experimentos propostos.
\item Um sistema de controle e navegação para o drone via rede IP.
\item Um drone capaz de realizar comunicação de dados numa rede IP.    
\end{itemize}

\section{Trabalhos correlacionados}

 - Artigos de Drones LTE
 \cite{artigo_relacionado_1}

\section{Organização do documento}
Resumir os seguintes pontos:
\begin{enumerate}
    \item Introducao teorica
    \item Definições
    \item Arquitetura Proposta
    \item Metodologia
\end{enumerate}



%

\chapter{Introdução teórica}
%
% retira numeracao da pagina, conforme as normas de apresentacao.
\thispagestyle{empty} 
%


\section{Definições iniciais}

\subsection{Drones e VANTs}
Drones são sistemas aéreos automatizados que incluem tanto os VANTs - Veículos Aéreos Não Tripulados, capazes de voar por milhares de 
kilometros, quanto drones pequenos que voam em locais confinados. São veículos aéreos não tripulados, automatizados ou controlados 
remotamente, podem carregar diversos tipos de carga e fazer vários tipos de missões.\\
Devido a avanços nas tecnologias envolvidas na fabricação, nos sistemas de navegação, nos sensores e sistemas de armazenamento de 
energia, uma grande variedade de tipos de drones surgiram nas últimas décadas. 

\subsection{Classificação de VANTs}

\begin{figure}[!htbp]
  \centering
  \includegraphics[width=0.7\textwidth]{Images/introducao/drone_classification.png}
  \caption{Classificação de VANTs por tipo de pouso, aerodinâmica e quantidade de rotores.}
  \label{fig:drone_classification.png}
\end{figure}

Como mostrado no organograma da figura a cima~\ref{fig:drone_classification.png}, as categorizações mais comuns para os VANTs são obtidas pelos atributos "tipo de pouso", "aerodinâmica" e "quantidade de rotores".

O tipo de aerodinâmica responsável pela sustentação das aeronaves no ar é bastante diversificado. Uma categoria bem comum de drones sa VANTs de asa fixa, que semelhantemente aos aviões, tiram proveito da geometria das asas e da velocidade gerada por um ou mais motores para obterem sustentação. 

A classificaçao de aterrisagem, com a nomenclatura VTOL e HTOL diz respeito apenas aos VANTs de asa fixa visto que apenas esses tipos de VANTs tiram proveito tanto da aterrisagem vertical quanto da autonomia proveniente do vôo horizontal com asas fixas. 

\subsection{Componentes dos VANTs de Asa Rotativa}

\subsection{Controladora PixHawk}

\subsection{UART}

\subsection{Telemetria}

\subsection{Protocolo Mavlink}

\subsection{Computadores de Bordo}

\subsection{ROS - Robot Operative System}

\subsection{Drones DJI}

\section{Arquitetura do sistema proposto}

Explicação dos componentes da arquitetura do sistema proposto.
%
%\begin{figure}[ht]
%    \centering
%    \includegraphics{Images/introducao/planta_inicial.png}
%    \caption{Diagrama em blocos da planta didática em configuração original.}
%    \label{fig:planta_inicial}
%\end{figure}


%%%%%%%%%%%%%%%%%%%%%%%%%%%%%%%%
%%%%%%%% Metodologia %%%%%%%%%%%
%%%%%%%%%%%%%%%%%%%%%%%%%%%%%%%%

\chapter{Metodologia}
\label{chapter:Metodologia}
%
% retira numeracao da pagina, conforme as normas de apresentacao.
\thispagestyle{empty} 
%
\section{Pesquisa}


%
\section{Objetivos do método aplicado}
A metodologia aplicada possui objetivo descritivo e exploratório. Com o sistema devidamente construído, faremos experimentos que comprovam algumas possibilidades de uso para o equipamento.

\section{Abordagem}
adotar-se-á uma análise qualitativa dos resultados obtidos a partir da implementação do novo sistema, em observância aos seguintes questionamentos:
\begin{enumerate}
    \item Com o sistema desenvolvido será possível ao professor, operador do sistema. controlar o experimento remotamente?
    \item Com o sistema desenvolvido a turma da disciplina conseguirá acompanhar o experimento em um ambiente diferente do laboratório?
    \item pergunta avaliativa 3
    \item pergunta avaliativa 4
\end{enumerate}
%
%%%%%%%%%%%%%%%%
%

\section{Descrição do problema}


%%%%%%%%%%%%%%%%%%%%%%%%%%%%%%%%%%%%%%%%%%%%%%%%%%%%%%%
%
\chapter{Detalhamento da solução}
%
% retira numeracao da pagina, conforme as normas de apresentacao.
\thispagestyle{empty} 
%


\section {Série de computadores Raspberry Pi}
%
O Raspberry Pi é uma série de computadores de placa única e tamanho reduzido, que recebem a denominação  SoC (\textit{System on Chip})~\cite{url:soc} à qual são conectados os seguintes dispositivos: monitor, teclado, e mouse. 

Desenvolvido no Reino Unido pela \href{https://www.raspberrypi.org/}{Raspberry Pi Foundation} tendo, como principais objetivos, contribuir para inclusão digital, promoção de ensino básico em ciência da computação e empoderamento social. Uma alternativa de ensino com baixo custo para escolas e estudantes~\cite{url:raspberry_wiki}. A Tabela~\ref{tab:Especificacoes} apresenta as especificações técnicas do modelo Raspberry Pi Revision 2 - Element 14, utilizado pelo grupo PET-Tele. 
%
\begin{table}[!htbp]
    \centering
    \begin{tabular}{ |c|c| } 
        \hline
        Chip & Broadcom BCM2835 SoC Full HD Processador de Aplicações Multimídia\\
        \hline
        CPU & 700 Mhz ARM1176JZ-F Processador de Baixa Potência de aplicações \\
        \hline
        GPU & Dois Núcleos, VideoCore IV, Co-Processador de Multimídia \\ 
        \hline
        Memória & 512MB SDRAM  \\
        \hline
        Ethernet & Onboard 10/100 Ethernet com conector RJ-45 \\
        \hline
        USB 2.0 & Dois Conectores USB 2.0 \\ 
        \hline
        Saída de Vídeo & HDMI e Composição RCA (PAL e NTSC) \\ 
        \hline
        Saída de Áudio & Conector 3.5 mm ou HDMI \\
        \hline
        Armazenamento & SD, MMC, SDIO Card Slot \\
        \hline
        Dimensões & 8.6 cm X 5.4 cm X 1.7 cm \\ 
        \hline
    \end{tabular}
    \caption{Especificações técnicas do modelo Raspberry Pi Revision 2 - Element14.}
    \label{tab:Especificacoes}
\end{table}

\pagebreak

Na Figura~\ref{fig:rasp3b.jpg.0} é mostrado uma fotografia do modelo. 
Dentre os 25 Pinos presentes neste Raspberry, 17 podem ser usados como entradas ou saídas de uso geral, 5 como terminais comuns (GND), 2 como fontes de tensão de valor +5V e 2 como fontes de tensão de valor +3.3V. 

%
A Tabela~\ref{tab:Raspberry Pinout} reúne uma descrição de todos os pinos.
%
\begin{figure}[!htbp]
    \centering
    \includegraphics[width=0.7\textwidth]{Images/introducao/rasp3b.jpg}
    \caption{Raspberry Pi Revision 2.0.}
    \label{fig:rasp3b.jpg.0}
\end{figure}
%
\begin{table}[!htbp]
    \centering
    \begin{tabular}{|c|c|c|c|} 
         \hline
         \textbf{N} & \textbf{Descrição} & \textbf{N} & \textbf{Descrição} \\ [0.5ex]
         \hline
         1 & Saída de +3.3V & 14 & Terminal Comum GND\\
         \hline
         2 & Saída de +5V & 15 & GPIO22 \\
         \hline
         3 & GPIO2 | Pull-Up | I2C-SDA & 16 & GPIO23 \\ 
         \hline
         4 & Saída de +5V & 17 & 3V3 \\
         \hline
         5 & GPIO3 | Pull-Up | I2C-SCE & 18 & GPIO24 \\
         \hline
         6 & Terminal Comum GND & 19 & GPIO10 | SPI | MOSI \\ 
         \hline
         7 & GPIO4 & 20 & Terminal Comum GND \\ 
         \hline
         8 & GPIO14 | UART | TXD & 21 & GPIO9 | SPI | MISO\\
         \hline
         9 & Terminal Comum GND & 22 & GPIO25\\
         \hline
         10 & GPIO15 | UART | RXD & 23 & GPIO11 | SPI | CLK\\ 
         \hline
         11 & GPIO17 | UART-RTS & 24 & GPIO8 | SPI | CE0\\
         \hline
         12 & GPIO18 | PWM & 25 & Terminal Comum GND\\ 
         \hline
         13 & GPIO27 & 26 & GPIO7 | SPI | CE1\\ 
         \hline
    \end{tabular}
    \caption{Descrição dos pinos do Raspberry Pi Rev 2.0 - Element 14.}
    \label{tab:Raspberry Pinout}
\end{table}





\chapter{Resultados}
%
% retira numeracao da pagina, conforme as normas de apresentacao.
\thispagestyle{empty} 
%
apresentação de resultados (numéricos e/ou gráficos) 
de cálculos e/ou de simulações, requeridos na especificação do trabalho.

%%%%%%%%%%%%%%%%%%%%%%%%%%%%%%%%%%%%%%%%%%%%%%%%%%%


%%%%%%%%%%%%%%%%%%%%%%%%%%%%%%%%%%%%%%%%%%%%%%%%%%%%%%%%
%                      Conclusão                       %
%%%%%%%%%%%%%%%%%%%%%%%%%%%%%%%%%%%%%%%%%%%%%%%%%%%%%%%%

\chapter{Conclusão}
%
% retira numeracao da pagina, conforme as normas de apresentacao.
\thispagestyle{empty} 
%

\hl{Hipótese testada
Se realizarmos o experimento em outro ambiente com maior capacidade para que todos possam assistir de forma homogênea atenderemos a demanda da turma, teremos um aprendizado mais efetivo e melhoria no tempo dedicado a esse experimento.

Responda as perguntas avaliativas para definir a conclusão}
\begin{enumerate}
    \item Com o sistema desenvolvido será possível ao professor usuário controlar o experimento proposto remotamente?
    \item pergunta avaliativa 2
    \item pergunta avaliativa 3
    \item pergunta avaliativa 4
\end{enumerate}

A presente monografia detalha o processo de pesquisa, projeto e construção de um sistema para atuação em uma planta didática localizada no Laboratório de Drenagem Irrigação e Saneamento~\cite{url:ladisan}.

Utilizando, como base, o documento 
\textbf{Uso de Sensores e Acesso Remoto para a Realização de Aula Prática Sobre Reservatórios de Detenção Aplicados à Drenagem urbana}~\cite{article:tcc_lorraine_maria} 
%
e o artigo 
\textbf{Realização de Aula Prática Remota a Partir de Laboratório Equipado com Modelo Físico Sobre Detenção de Água de Chuva}~\cite{article:lorraine_maria}, 
publicado no COBENGE 2018~\cite{url:cobenge}, 
%
foi desenvolvido um conjunto de melhorias e novos processos que permitem monitoramento e maior e melhor atuação do ``usuário'' (computador remoto localizado na sala de aula) na planta didática operando no Laboratório LaDISan~\cite{url:ladisan}. 
%
O conjunto de melhorias compreende uma câmera com controle remoto, um servomotor acoplado a um registro de admissão de água, uma interface de acesso e um dispositivo Raspberry Pi operando como controlador do servomotor e computador auxiliar para o Arduino. 

Os resultados foram satisfatórios no que tange aos objetivos propostos. 
Foram realizadas duas aulas no ano de 2019, uma no primeiro semestre letivo e outra no segundo semestre.

O sistema ainda não contempla uma solução fechada pronta para comercialização conforme foi citado no Capítulo~\ref{chapter:Metodologia}.
%
Ainda se faz necessário melhorias e correções que podem ser implementadas conforme novas demandas levantadas por alunos da disciplina e o professor Dario Sobrenome Sobrenome.
%
As demandas podem ser implementadas pelos futuros bolsistas do grupo PET-Tele. Complementarmente a este documento existe um relatório técnico com detalhamento das etapas de construção, disponível no \textit{website} do Grupo PET-Tele~\cite{url:projeto_ladisan}

%%%%%%%%%%%%%%%%%%%%%%%%%%%%%%%%%%%%%%%%%%%%%%%%%%%%%%%%
%          Sugestoes para trabalhos futuros            %
%%%%%%%%%%%%%%%%%%%%%%%%%%%%%%%%%%%%%%%%%%%%%%%%%%%%%%%%

\chapter{Sugestões para trabalhos futuros}
%
% retira numeracao da pagina, conforme as normas de apresentacao.
\thispagestyle{empty} 
%
%
Em observância ao caráter amplo e diverso dos conceitos aqui utilizados, diversas vertentes de trabalhos futuros podem ser identificadas. Tais vertentes, assim como os trabalhos individuais em cada uma das vertentes, podem ser listados e resumidos.
%
\begin{enumerate}
    \item Vertente Técnica Funcional
        \begin{enumerate}
            \item Novas interfaces de usuário com mais opções
            \item Desenvolvimento de novas curvas hidrográficas
        \end{enumerate}
    \item Vertente Facilidade de Uso
        \begin{enumerate}
            \item Encapsulamento de hardware em produto final
            \item Simplificação do configuração e uso para usuário final.
        \end{enumerate}
    \item Vertente casos de testes
        \begin{enumerate}
            \item Coleta de indicadores para medir o grau de satisfação da turma com o sistema.
            \item Aumentar o número de testes a fim de identificar falhas.
        \end{enumerate}
\end{enumerate}

Vale ressaltar dentro dessas vertentes a necessidade de realização de mais testes com o sistema em operação como forma de garantir que as funcionalidades propostas estejam com o funcionamento correto.]

Para isso, é de fundamental importância a elaboração de novos casos de uso e simulações teste, bem como a implementação de atualizações otimizadas dos programas desenvolvidos. Uma pesquisa avaliando os parâmetros funcionais qualitativos da interface proposta na Seção~\ref{subsec:nova_interface} pode servir de suporte para as exigências aqui formuladas.

%%%%%%%%%%%%%%%%%%%%%%%%%%%%%%%%%%%%%%%%%%%%%%%%%%%%%%


%%%%%%%%%%%%%%%%%%%%%%%%%%%%%%%%%%%%%%%%%%%%%%%%%%%%%%%%%%%%%%%%%%%
%                  Referencias Bibliograficas                     % 
%%%%%%%%%%%%%%%%%%%%%%%%%%%%%%%%%%%%%%%%%%%%%%%%%%%%%%%%%%%%%%%%%%%

\bibliographystyle{apalike}
%
\bibliography{./MyBibFiles/my_refs}
%
\addcontentsline{toc}{chapter}{\refname}
%
\thispagestyle{myheadings}

%
%%%%%%%%%%%%%%%%%%%%%%%%%%%%%%%%%%%%%%%%%%%%%%%%%%%%

%
%%%%%%%%%%%%%%%%%%%%%%%%%%%%%%%%%%%%%%%%%%%%%%%%
%
%
% Apendices e anexos
%
\appendix
%
%
%%%%%%%%%%%%%%%%%%%%%%%%%%%%%%%%%%%%%%%%%%%%%%%%%%%%%%%%
%
\chapter{Apêndice 1}
%
% retira numeracao da pagina, conforme as normas de apresentacao.
\thispagestyle{empty} 
%
Segue abaixo um guia de instalação completo dos \textit{Softwares} de simulação e suas respectivas dependências para Ubuntu 20.04 LTS.

\section{Intalando o ArduPilot e MavProxy}

\subsection{Clonando o repositório \textit{git} para sua máquina}
\begin{lstlisting}[language=bash]
  $ cd ~
  $ sudo apt install git
  $ git clone https://github.com/ArduPilot/ardupilot.git
\end{lstlisting}

\subsection{Instalando as dependências e recompilando o perfil}
\begin{lstlisting}[language=bash]
  $ cd ardupilot/Tools/environment_install/install-prereqs-ubuntu.sh -y
  $ . ~/.profile
\end{lstlisting}

\subsection{Mudando para a \textit{branch} do ArduCopter}
\begin{lstlisting}[language=bash]
  $ git checkout Copter-4.0.4
  $ git submodule update --init --recursive
\end{lstlisting}

\subsection{Rodando SITL (\textit{Software In The Loop})}
\begin{lstlisting}[language=bash]
  $ cd ~/ardupilot/ArduCopter
  $ sim_vehicle.py -w
\end{lstlisting}

\section{Instalando o Gazebo e o \textit{plugin} do ArduPilot}

\subsection{Atualizando a lista de fontes para \textit{download}}
\begin{lstlisting}[language=bash]
  $ sudo sh -c 'echo "deb http://packages.osrfoundation.org/gazebo/ubuntu-stable `lsb_release -cs` main" > /etc/apt/sources.list.d/gazebo-stable.list'
  $ wget http://packages.osrfoundation.org/gazebo.key -O - | sudo apt-key add -
  $ sudo apt update
\end{lstlisting}

\subsection{Instalando o \textit{plugin} do Gazebo para comunicação com o ArduPilot}
\begin{lstlisting}[language=bash]
  $ cd ~
  $ git clone https://github.com/khancyr/ardupilot_gazebo.git
  $ cd ardupilot_gazebo
  $ mkdir build
  $ cd build
  $ cmake ..
  $ make -j4
  $ sudo make install
  $ echo 'source /usr/share/gazebo/setup.sh' >> ~/.bashrc
  $ echo 'export GAZEBO_MODEL_PATH=~/ardupilot_gazebo/models' >> ~/.bashrc
  $ . ~/.bashrc
\end{lstlisting}

\subsection{Executando a simulação}
\begin{lstlisting}[language=bash]
  No primeiro terminal do linux rode o Gazebo: 
  $ gazebo --verbose ~/ardupilot_gazebo/worlds/iris_arducopter_runway.world
  
  No segundo terminal do linux rode o SITL:
  $ cd ~/ardupilot/ArduCopter/
  $ sim_vehicle.py -v ArduCopter -f gazebo-iris --console
\end{lstlisting}

\section{Instalando o \textit{ROS} e o configurando o \textit{Catkin}}

\subsection{Atualizando a lista de fontes para \textit{download}}
\begin{lstlisting}[language=bash] 
  $ sudo sh -c 'echo "deb http://packages.ros.org/ros/ubuntu $(lsb_release -sc) main" > /etc/apt/sources.list.d/ros-latest.list'
  $ sudo apt install curl
  $ curl -s https://raw.githubusercontent.com/ros/rosdistro/master/ros.asc | sudo apt-key add -
  $ sudo apt update
\end{lstlisting}

\subsection{Instalando o \textit{ROS}}
\begin{lstlisting}[language=bash] 
  $ sudo apt install ros-noetic-desktop-full
\end{lstlisting}

\subsection{Configurando o ambiente}
\begin{lstlisting}[language=bash] 
  $ source /opt/ros/noetic/setup.bash
  $ echo "source /opt/ros/noetic/setup.bash" >> ~/.bashrc
  $ source ~/.bashrc
  $ echo "source /opt/ros/noetic/setup.zsh" >> ~/.zshrc
  $ source ~/.zshrc
\end{lstlisting}

\subsection{Instalando as dependências para os pacotes dos \textit{ROS}}
\begin{lstlisting}[language=bash] 
  $ sudo apt install python3-rosdep python3-rosinstall python3-rosinstall-generator python3-wstool build-essential
  $ sudo apt install python3-rosdep
  $ sudo rosdep init
  $ rosdep update
\end{lstlisting}

\subsection{Configurando o \textit{Catkin}}
\begin{lstlisting}[language=bash] 
  $ sudo apt-get install python3-wstool python3-rosinstall-generator python3-catkin-lint python3-pip python3-catkin-tools
  $ pip3 install osrf-pycommon
  $ mkdir -p ~/catkin_ws/src
  $ cd ~/catkin_ws
  $ catkin init
\end{lstlisting}

\subsection{Instalando as dependências para o \textit{Catkin}}
\begin{lstlisting}[language=bash] 
  $ sudo apt-get install python3-wstool python3-rosinstall-generator python3-catkin-lint python3-pip python3-catkin-tools
  $ pip3 install osrf-pycommon
  $ mkdir -p ~/catkin_ws/src
  $ cd ~/catkin_ws
  $ catkin init
\end{lstlisting}

\subsection{Instalando as \textit{MAVROS} e \textit{MAVLink}}
\begin{lstlisting}[language=bash] 
  $ cd ~/catkin_ws
  $ wstool init ~/catkin_ws/src
  $ rosinstall_generator --upstream mavros | tee /tmp/mavros.rosinstall
  $ rosinstall_generator mavlink | tee -a /tmp/mavros.rosinstall
  $ wstool merge -t src /tmp/mavros.rosinstall
  $ wstool update -t src
  $ rosdep install --from-paths src --ignore-src --rosdistro `echo $ROS_DISTRO' -y
  $ catkin build
\end{lstlisting}

\subsection{Atualizando o arquivo \textit{.bachrc}}
\begin{lstlisting}[language=bash] 
  $ echo "source ~/catkin_ws/devel/setup.bash" >> ~/.bashrc
  $ source ~/.bashrc
\end{lstlisting}

\section{Instalando as dependências geográficas e clonando o repositório de simulação do \textit{Intelligent Quads}}

\subsection{Instalando as dependências geográficas}
\begin{lstlisting}[language=bash] 
  $ sudo ~/catkin_ws/src/mavros/mavros/scripts/install_geographiclib_datasets.sh]
\end{lstlisting}

\subsection{Clonando pacote ROS de simulação do \textit{Intelligent Quads}}
\begin{lstlisting}[language=bash] 
  $ cd ~/catkin_ws/src
  $ git clone https://github.com/Intelligent-Quads/iq_sim.git
  $echo "GAZEBO_MODEL_PATH=${GAZEBO_MODEL_PATH}:$HOME/catkin_ws/src/iq_sim/models" >> ~/.bashrc
  $ cd ~/catkin_ws
  $ catkin build
  $ source ~/.bashrc
\end{lstlisting}

\section{Instalando o \textit{QGround Control}}

\subsection{Alterando permissões e instalando o \textit{QGround Control}}
\begin{lstlisting}[language=bash] 
  $ sudo usermod -a -G dialout $USER
  $ sudo apt-get remove modemmanager
  $ wget https://s3-us-west-2.amazonaws.com/qgroundcontrol/latest/QGroundControl.AppImage
  $ chmod +x ./QGroundControl.AppImage 
  $ ./QGroundControl.AppImage
\end{lstlisting}
%
%%%%%%%%%%%%%%%%%%%%%%%%%%%%%%%%%%%%%%%%%%%%%%%%%%%%%%%%
%
\end{document}
%


%%%%%%%%%%%%%%%%%%%%%%%%%%%%%%%%%%%%%%%%
%     Insercao do Indice Remissivo     %
%%%%%%%%%%%%%%%%%%%%%%%%%%%%%%%%%%%%%%%%
%
% Makeindex Database 
% (baseada nos arquivos: XXX.idx --> xxx.ind --> xxx.ilg)
%
\printindex
%
\addcontentsline{toc}{chapter}{\indexname}
%
\thispagestyle{myheadings}


%%%%%%%%%%%%%%%%%%%%%%%%%%%%%%%%%%%%%%%%%%%%%%%%%%%%%%


%
\end{document}
%

%%%%%%%%%%%%%%%%%
% Fim do modelo %
%%%%%%%%%%%%%%%%%
