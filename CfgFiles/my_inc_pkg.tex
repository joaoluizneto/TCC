%
%
%%%%%%%%%%%%%%%%%%%%
%   Pacotes gerais %
%%%%%%%%%%%%%%%%%%%%
%
%
% Padrao brasileiro:
%   a primeira frase eh indentada
%   para todos os paragrafos.
%
\usepackage{indentfirst}
%
%
% Pacotes para lingua portuguesa
%
%%%\usepackage{babel}
%%%\usepackage[brazil]{babel}
\usepackage[brazilian]{babel}
%%%\usepackage[portuguese]{babel}
%%%\babelprovide[import=pt-BR,main]{portuguese}
%
%
%%%\usepackage{ae}
\usepackage{lmodern}
\usepackage[T1]{fontenc}
%
%
% Padrao de codificacao 
% dos caracteres no arquivo '.tex'
%
%%%\usepackage{inputenc}
%\usepackage[ansinew]{inputenc}
\usepackage[utf8]{inputenc}
%
%
% Pacotes matematicos
%
% Package dependency hierarchy 
% in terms of the AMS-LaTeX bundle:
%
% - amsmath (miscellaneous enhancements)
%   - amstext (text embedded in mathematics)
%     - amsgen (*)
%
% - amsbsy (bold symbols)
%   - amsgen (*)
%
% - amsopn (operator name commands)
%   - amsgen (*)
%
% - amssymb (extended symbol collection)
%   - amsfonts (*)
%
% - amsthm (theorem-like environments)
%
% (*) This package has no dependencies.
%
\usepackage{amsmath}
\usepackage{amsfonts}
\usepackage{amssymb}

%
%
% Pacotes graficos
%
%\usepackage{graphics}
\usepackage{graphicx}
\usepackage{subfig}
%\usepackage{subfigure}
\usepackage{epsfig}
%\usepackage{rotating}
%
%
% Pacotes sobre tabelas
%
\usepackage{multirow}
%
%
% Pacote para fazer 'Indice Remissivo'
%
%%%\usepackage{makeidx}
\usepackage{imakeidx}
\makeindex[intoc]
%
%
% Pacotes uteis para revisao do texto
%
% Comentarios de varias linhas
%
\usepackage{comment}
%
%
%%%%%%%%%%%%%%%%%%%%%
% Incluir Códigos   %
%%%%%%%%%%%%%%%%%%%%%
\usepackage{listings}
%
%% Configuracoes do pacote listings
\lstdefinestyle{mystyle}{
language=Python,
basicstyle=\ttm,
% Add keywords here
morekeywords={self, if, True, False, for, except, raise, return, try},
keywordstyle=\ttb\color{deepblue},
% Custom highlighting
emph={set_mode, set_servo_pulsewidth, sleep, OUTPUT, stop, exit},          
% Custom highlighting style
emphstyle=\ttb\color{deepred},    
stringstyle=\color{deepgreen},
% Any extra options here
frame=single,                       
showstringspaces=false,
commentstyle=\color{green}
}
% 
\lstset{
style=mystyle,
breaklines=true
}
%
%
% definicao de comando para Lista de Códigos
%
% Listing -> Código
\renewcommand{\lstlistingname}{Código}
% List of Listings -> Lista de Códigos
\renewcommand{\lstlistlistingname}{Lista de \lstlistingname s}
%
%% PACOTES QUE SERÃO REMOVIDOS APÓS REVISÃO
% Realce de texto
\usepackage{soulutf8}
\usepackage{soul}
% Extensão de letras gregas (Vou tentar Excluir isso)
\usepackage{upgreek}

%%%%%%%%%%%%%%%%% FIM DESSES PACOTES %%%%%%%%%%%%%%

% Marcacao de texto com cor
%
% Explicar...
%
%%%\usepackage{color}
%\usepackage{xcolor}
\usepackage[table,xcdraw]{xcolor}
%
\definecolor{red}  {rgb}{1,0,0}
\definecolor{green}{rgb}{0,1,0}
\definecolor{blue} {rgb}{0,0,1}
\definecolor{deepblue}{rgb}{0,0,0.5}
\definecolor{deepred}{rgb}{0.6,0,0}
\definecolor{deepgreen}{rgb}{0,0.5,0}
\definecolor{green}{rgb}{0,0.8,0}
%
\DeclareFixedFont{\ttb}{T1}{txtt}{bx}{n}{12} % for bold
\DeclareFixedFont{\ttm}{T1}{txtt}{m}{n}{12}  % for normal
%
%
% O pacote 'xurl' 
% faz quebra automatica de linha 
% para um URL muito grande...
%
%%%\usepackage{url}
\usepackage{xurl}
%
\usepackage{hyperref}
%
\usepackage{float}
%
% Explicar e posicionar...
%
%%%\usepackage{memhfixc}
%

%%%%%%%%%%%%%%%%%%%%%%%%%%%%%%%%%%%%%%%%%%%%%%%%%%%%%%